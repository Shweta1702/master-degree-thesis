\documentclass[a4paper,10pt]{memoir}
\usepackage[english]{babel}
\usepackage{wrapfig}
\usepackage[pdftex]{graphicx}
\usepackage{graphviz}
\usepackage{graphicx}
\usepackage{listingsutf8}
\usepackage{amsmath}
\usepackage{afterpage}
\usepackage{subfig}
\usepackage{quoting, lipsum}
\usepackage{float}

%Define the listing package
\usepackage{listings} %code highlighter
\usepackage{color} %use color
\definecolor{mygreen}{rgb}{0,0.6,0}
\definecolor{mygray}{rgb}{0.5,0.5,0.5}
\definecolor{mymauve}{rgb}{0.58,0,0.82}
 
%Customize a bit the look
\lstset{ %
backgroundcolor=\color{white}, % choose the background color; you must add \usepackage{color} or \usepackage{xcolor}
basicstyle=\footnotesize, % the size of the fonts that are used for the code
breakatwhitespace=false, % sets if automatic breaks should only happen at whitespace
breaklines=true, % sets automatic line breaking
captionpos=b, % sets the caption-position to bottom
commentstyle=\color{mygreen}, % comment style
deletekeywords={...}, % if you want to delete keywords from the given language
escapeinside={\%*}{*)}, % if you want to add LaTeX within your code
extendedchars=true, % lets you use non-ASCII characters; for 8-bits encodings only, does not work with UTF-8
frame=single, % adds a frame around the code
keepspaces=true, % keeps spaces in text, useful for keeping indentation of code (possibly needs columns=flexible)
keywordstyle=\color{blue}, % keyword style
% language=Octave, % the language of the code
morekeywords={*,...}, % if you want to add more keywords to the set
numbers=left, % where to put the line-numbers; possible values are (none, left, right)
numbersep=5pt, % how far the line-numbers are from the code
numberstyle=\tiny\color{mygray}, % the style that is used for the line-numbers
rulecolor=\color{black}, % if not set, the frame-color may be changed on line-breaks within not-black text (e.g. comments (green here))
showspaces=false, % show spaces everywhere adding particular underscores; it overrides 'showstringspaces'
showstringspaces=false, % underline spaces within strings only
showtabs=false, % show tabs within strings adding particular underscores
stepnumber=1, % the step between two line-numbers. If it's 1, each line will be numbered
stringstyle=\color{mymauve}, % string literal style
tabsize=2, % sets default tabsize to 2 spaces
title=\lstname % show the filename of files included with \lstinputlisting; also try caption instead of title
}
%END of listing package%
 
\definecolor{darkgray}{rgb}{.4,.4,.4}
\definecolor{purple}{rgb}{0.65, 0.12, 0.82}
 
%define Javascript language
\lstdefinelanguage{JavaScript}{
keywords={typeof, new, true, false, catch, function, return, null, catch, switch, var, if, in, while, do, else, case, break},
keywordstyle=\color{blue}\bfseries,
ndkeywords={class, export, boolean, throw, implements, import, this},
ndkeywordstyle=\color{darkgray}\bfseries,
identifierstyle=\color{black},
sensitive=false,
comment=[l]{//},
morecomment=[s]{/*}{*/},
commentstyle=\color{purple}\ttfamily,
stringstyle=\color{red}\ttfamily,
morestring=[b]',
morestring=[b]"
}

\lstdefinelanguage{docker}{
  keywords={FROM, RUN, COPY, ADD, ENTRYPOINT, CMD,  ENV, ARG, WORKDIR, EXPOSE, LABEL, USER, VOLUME, STOPSIGNAL, ONBUILD, MAINTAINER},
  keywordstyle=\color{blue}\bfseries,
  identifierstyle=\color{black},
  sensitive=false,
  comment=[l]{\#},
  commentstyle=\color{purple}\ttfamily,
  stringstyle=\color{red}\ttfamily,
  morestring=[b]',
  morestring=[b]"
}

\lstdefinelanguage{docker-compose-2}{
  keywords={version, volumes, services},
  keywordstyle=\color{blue}\bfseries,
  keywords=[2]{image, environment, ports, container_name, ports, links, build},
  keywordstyle=[2]\color{blue}\bfseries,
  identifierstyle=\color{black},
  sensitive=false,
  comment=[l]{\#},
  commentstyle=\color{purple}\ttfamily,
  stringstyle=\color{red}\ttfamily,
  morestring=[b]',
  morestring=[b]"
}

\lstset{
language=JavaScript,
extendedchars=true,
basicstyle=\footnotesize\ttfamily,
showstringspaces=false,
showspaces=false,
numbers=left,
numberstyle=\footnotesize,
numbersep=9pt,
tabsize=2,
breaklines=true,
showtabs=false,
captionpos=b
}

\newcommand\blankpage{%
    \null
    \thispagestyle{empty}%
    \addtocounter{page}{-1}%
    \newpage}

\usepackage[chapter]{minted}
\usepackage{adjustbox}
\usepackage{hyperref}
\hypersetup{
  colorlinks   = true,    % Colours links instead of ugly boxes
  urlcolor     = blue,    % Colour for external hyperlinks
  linkcolor    = black,    % Colour of internal links
  citecolor    = black      % Colour of citations
}

% import package
\usepackage{FrontespizioSapienza}

\pagestyle{plain}%%to insert the number of the page

% declare info
\FSSTitolo{Title}
\FSSFacolta{Ingegneria dell'Informazione, Informatica e Statistica}
\FSSCorso{Cybersecurity}

\FSSCandidato{Edoardo Ottavianelli}
\FSSMatricola{1756005}
\FSSRelatore{Prof. Marco Polverini}
\FSSAnnoAccademico{2022/2023}


\begin{document}

\frontmatter


% print title
\maketitle
\cleardoublepage

%\vspace*{10cm}
%\begin{flushright}
%\textsl{...}
%\end{flushright}
%\cleardoublepage

% ======================================= ABSTRACT ================================================
\begin{abstract}
ABSTRACT

\end{abstract}
\cleardoublepage

\tableofcontents
\cleardoublepage

\mainmatter

\renewcommand\chapterheadstart{}
\renewcommand\printchaptername{}
\renewcommand\chapternamenum{}
\renewcommand\printchapternum{}
\renewcommand\afterchapternum{}
\renewcommand\printchaptertitle[1]{\chaptitlefont \thechapter. \space #1}


% ======================================= CHAPTER 1 ================================================
\chapter{Introduction, Problems and Goals}

\section{Introduction to ONOS}

\clearpage

\section{Introduction to CAP attacks}

\clearpage

\section{Existing solutions}

\clearpage

% ======================================= CHAPTER 2 ================================================
\chapter{Architecture design}

\section{ONOS internals}

\clearpage

\section{ApplicationKeyStore}

\clearpage

% ======================================= CHAPTER 3 ================================================
\chapter{Implementation}

\section{Authentication}

\clearpage

\section{Logs}

\clearpage

\section{Data Mining}

\subsection{Logistic Regression}

\subsection{K nearest neighbours}

\subsection{Models' score}

\clearpage

\section{Log Analysis}

\subsection{Build App-Store interactions graph}

\subsection{Find CAP gadgets in graph}

\subsection{Find CAP gadgets in log file}

\clearpage

% ======================================= CHAPTER 4 ================================================
\chapter{Security}

\section{Proposed solution Security}

\subsection{Key generation}

Is the key secure? Y

\subsection{Key distribution}

Is the key distribution secure? Y


\subsection{Key Storage}

Is the key storage secure? Y unless reflection and RAM


\subsection{Legit App Security}

What about key security from legit apps perspective? Idc

\clearpage

\section{Vulnerability research}

\subsection{Metodology}

\subsection{CVE-2023-24279}

\subsection{Failed attempts}

\clearpage

\section{Web targeted CAP attack demonstration}

\clearpage

% ======================================= CHAPTER 5 ================================================
\chapter{Tests, conclusions and challenges}

\section{Tests}

\clearpage

\section{Conclusions}

\clearpage

\section{Future challenges}

\clearpage

\refstepcounter{chapter}

% ======================================= CHAPTER 6 ================================================

\chapter*{Thanks}

\refstepcounter{chapter}

% ======================================= BIBLIOGRAPHY ================================================
\begin{thebibliography}{9}

\bibitem{terra}
  \textbf{The inner structure of the Earth}.\\
  \href{http://sci.fgt.bme.hu/~volgyesi/gravity/ppfold.pdf}{http://sci.fgt.bme.hu/~volgyesi/gravity/ppfold.pdf}\\
  1982, L.Volgeyesi, M. Moser

\end{thebibliography}

\end{document}